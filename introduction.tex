\documentclass[memoire.tex]{subfiles}
\begin{document}
\newpage
\chapter*{Introduction}
Afin d'orienter les étudiants arrivant en première année de licence vers un parcours qui leur semble le plus adapté, il serait intéressant de pouvoir leur fournir un
aperçu des chemins suivis par leur prédécesseur et éventuellement le poste auquel
ceux-ci travaillent désormais. C'est pourquoi ce document s'oriente vers la modélisation de carrière des étudiants et leur catégorisation à l'aide de méthodes
de clustering. Le premier obstacle étant la catégorisation des métiers qui sont le plus
souvent vagues et de certains titre tel que consultant pouvant couvrir une grande
variété de professions. Par conséquent, la première partie de mon mémoire se concentrerait sur les différentes méthodes de clustering, ce qui permettrait de pouvoir affecter un tag aux différents métiers occupés par les anciens étudiants. Dans un second temps, les algorithmes utilisables seront étudiés suivis par les multiples types de cluster existant dans le soucis de choisir une structure adaptée à notre cas et ainsi obtenir une représentation adaptée des parcours d'étudiants.

\end{document}
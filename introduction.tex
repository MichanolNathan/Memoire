\documentclass[memoire.tex]{subfiles}
\begin{document}
\chapter*{Introduction}
Afin d'orienter les étudiants arrivants en première année de licence vers un parcours qui leur semble le plus adapté, il serait intéressant de pouvoir leur fournir un aperçu des chemins suivis par leur prédécesseur.\\
À cet effet, notre solution devra pouvoir répondre aux problématiques suivantes : 
\begin{itemize}
\item Comment catégoriser les différentes filières universitaires?
\item Quel secteur l'étudiant sera-t-il susceptible de pouvoir intégrer?
\item Fournir une modélisation de la trajectoire représentée par les différents parcours.
\end{itemize}
Afin de tenter d'apporter une réponse à ces deux premières problématiques, la première partie de ce mémoire se concentrera sur l'étude des différentes méthodes de clustering existantes afin de potentiellement apporter un premier élément de réponse concernant cette problématique de catégorisation des données. Dans la suite de ce document, seront étudiés les différents types de clusters existants que les différents algorithmes. Dans un second temps, nous étudierons les types de clusters existants ainsi que les différentes méthodes de clustering et algorithmes. Enfin, nous terminerons ce document par un chapitre de conclusion.\\

Accès GitHub : \begin{itemize}
\item \url{https://github.com/MichanolNathan/Memoire}
\end{itemize}
\end{document}
